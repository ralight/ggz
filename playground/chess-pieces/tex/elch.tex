% This is
%		elch.tex
% a plainTeX macro package to set fairy chess boards, with the
% elch-fonts.
%
% Copyright (C) 1989-92  by Elmar Bartel.
%
% This program is free software; you can redistribute it and/or modify
% it under the terms of the GNU General Public License as published by
% the Free Software Foundation; either version 1, or (at your option)
% any later version.
%
% This program is distributed in the hope that it will be useful,
% but WITHOUT ANY WARRANTY; without even the implied warranty of
% MERCHANTABILITY or FITNESS FOR A PARTICULAR PURPOSE.  See the
% GNU General Public License for more details.
%
% You should have received a copy of the GNU General Public License
% along with this program; if not, write to the Free Software
% Foundation, Inc., 675 Mass Ave, Cambridge, MA 02139, USA.


\font\chessfont=elch10
%
% this is for standalone applications
\font\numberfont=cmr10 scaled \magstep1
\font\authorfont=cmr9 scaled \magstep1
\font\origfont=cmsl8 scaled \magstep1
\font\dedicfont=cmsl8 scaled \magstep1
\font\stipfont=cmr9 scaled \magstep1
\font\remfont=cmr8 scaled \magstep1

%\def\numberfont{\tenpoint}%
%\def\authorfont{\ninepoint}%
%\def\origfont{\eightpoint\sl}%
%\def\dedicfont{\eightpoint\sl}%
%\def\stipfont{\ninepoint}%
%\def\remfont{\eightpoint}%
%
% Global counters
\newcount\boardnumber\boardnumber=1%
\newcount\solnumber\solnumber=1%
\newcount\whitecnt
\newcount\blackcnt
\newcount\neutrcnt

%
% BoardLocal registers
%\newcount\remcnt
%\newcount\rmc
%\newcount\bnumber
%
\countdef\remcnt=105
\countdef\rmc=106
\countdef\bnumber=107
%
%\dimendef\remwidth=10
\newdimen\remwidth
%\dimendef\itmwidth=11
\newdimen\itmwidth
%\dimendef\itmspace=12
\newdimen\itmspace
%\dimendef\remspace=13
\newdimen\remspace
%\dimendef\scratch=14
\newdimen\scratch
\newdimen\bdwidth
\newdimen\authorskip
\newdimen\remskip
%
%
\toksdef\numbertk=100
\toksdef\authortk=101
\toksdef\origtk=102
\toksdef\stiptk=104
\toksdef\remtk=105
\toksdef\dedictk=106
\toksdef\condtk=107
\toksdef\soltk=108
%
%\chardef\sqbx=10
%\chardef\stipbx=11
%\chardef\rembox=12
%\chardef\condbx=13
%\chardef\figcntbx=14
%
\def\setbdwidth{{\chessfont\global\bdwidth=8\fontdimen2\chessfont
\global\advance\bdwidth by.6pt}}%
\setbdwidth
%
\def\twelvechess{\font\chessfont=elch12 scaled \magstep1
\font\numberfont=cmr10 scaled \magstep1
\font\authorfont=cmr9 scaled \magstep1
\font\origfont=cmsl8 scaled \magstep1
\font\dedicfont=cmsl8 scaled \magstep1
\font\stipfont=cmr9 scaled \magstep1
\font\remfont=cmr8 scaled \magstep1
\authorskip=12pt
\remskip=11pt
\setbdwidth
}%
\def\elevenchess{\font\chessfont=elch11 scaled \magstep1
\font\numberfont=cmr10 scaled \magstep1
\font\authorfont=cmr9 scaled \magstep1
\font\origfont=cmsl8 scaled \magstep1
\font\dedicfont=cmsl8 scaled \magstep1
\font\stipfont=cmr9 scaled \magstep1
\font\remfont=cmr8 scaled \magstep1
\authorskip=11pt
\remskip=10pt
\setbdwidth
}%
\def\tenchess{\font\chessfont=elch10 scaled \magstep1
\font\numberfont=cmr9 scaled \magstep1
\font\authorfont=cmr8 scaled \magstep1
\font\origfont=cmsl8
\font\dedicfont=cmsl8
\font\stipfont=cmr8 scaled \magstep1
\font\remfont=cmr7 scaled \magstep1
\authorskip=9pt
\remskip=8pt
\setbdwidth
}%
%
\newbox\sqbx
\newbox\stipbx
\newbox\rembox
\newbox\condbx
\newbox\figcntbx
\newbox\authorbox
\newbox\origbox
\newbox\dedicbox
\newbox\tmpbx
%
\newif\ifnumber
\newif\ifauthor
\newif\ifpieces
\newif\iforig
\newif\ifstip
\newif\ifdedic
\newif\iffigcnt
\newif\ifgrid
\newif\ifcentered
\newif\ifcond
\newif\ifsolution
% Global ifs
\newif\ifalphanum\alphanumfalse
\newif\ifromannum\romannumfalse
\newif\iflowcasenum\lowcasenumfalse
%
\newwrite\solfd
\newwrite\etocfd
\def\clearsol{\immediate\openout\solfd=sol\the\solnumber.tex}\clearsol
\immediate\openout\etocfd=etoc.data
%
%\tracingmacros=1
%\tracingcommands=1
%
\def\bnum #1; {\numbertk={#1}\numbertrue}%
\def\ucnumber#1{\count0=#1\advance\count0 by `A\advance\count0 by -1\char\count0}%
\def\lcnumber#1{\count0=#1\advance\count0 by `a\advance\count0 by -1\char\count0}%
%
\def\eolist{\relax}%
\def\parse#1{\futurelet\next\checkit#1}%
\def\checkit{\ifx\next\eolist\let\todo=\relax\else\let\todo=\action\fi\todo}%
%
\def\setcentered#1, {\break\centerline{#1}\parse}%
\def\leftset{\par\hang\textindent}%
\def\setleftadj#1, {\leftset{\box0}#1\kern-5pt\parse}%
\def\author #1; {\authortrue
  \authortk={#1}%
  \ifcentered
    \let\action=\setcentered
    \setbox\authorbox=\vbox{\hsize=\bdwidth\noindent
      \authorfont\baselineskip=\authorskip\lineskip=1pt\lineskiplimit=.5pt\parskip=0pt
      \ifnumber\centerline{\numberfont\the\numbertk
      }\fi
      \authorfont
      \parse#1, \eolist}%
  \else
    \let\action=\setleftadj
    \setbox0=\hbox{\ifnumber\numberfont\ \the\numbertk)\fi}%
    \setbox\authorbox=\vbox{\hsize=\bdwidth\parindent=\wd0
      \authorfont\baselineskip=\authorskip\lineskip=1pt\lineskiplimit=.5pt\parskip=0pt
      \noindent\parse#1, \eolist}%
  \fi
}%
%
\def\orig #1; {\origtrue
  \ifcentered\let\action=\setcentered
  \else\let\action=\setleftadj\fi
  \setbox\origbox=\vbox{\vskip1pt\hsize=\bdwidth\parindent=0pt
    \origfont{\lineskip=1pt\baselineskip=0pt}%
    \parse#1, \eolist\vskip1pt}}%
%
\def\dedic #1; {\dedictrue
  \ifcentered\let\action=\setcentered
  \else\let\action=\setleftadj\fi
  \setbox\dedicbox=\vbox{\hsize=\bdwidth\parindent=0pt
    \dedicfont\lineskip=1pt\baselineskip=2pt%
    \parse#1, \eolist\vskip1pt}}%
%
\def\stip #1; {\stiptk={#1}\stiptrue}%
\def\cond #1; {\condtk={#1}\condtrue}%
\def\etoc #1; {\immediate\write\etocfd{\the\pknumber:\the\pageno:\the\boardnumber:#1}}%
%
\def\getmax#1/#2, {\setbox0=\hbox{#1}\ifdim\wd0>\parindent\parindent=\wd0\fi\parse}%
\def\setrem#1/#2, {\leftset{#1}#2\parse}%
\def\rem #1; {%
  \let\action=\getmax\parindent=0pt
  \parse#1, \eolist
  \ifdim\parindent>0pt\advance\parindent by .5em\fi
  \setbox\rembox=\vbox{\hsize=\bdwidth
    \tolerance=5000\emergencystretch=10pt
    \remfont\parskip=.5pt{\lineskip=1pt\baselineskip=\remskip\lineskiplimit=.5pt}
    \let\action=\setrem
    \parse#1, \eolist}}%
%
\def\initvars{%
	\soltk={}%
	\remcnt=20\global\whitecnt=0\global\blackcnt=0\global\neutrcnt=0
	\remwidth=0pt\itmwidth=0pt
	\bnumber=\boardnumber
}%
%
\def\venbox#1{\vbox{\hrule\hbox{\vrule{#1}\vrule}\hrule}}%
\def\henbox#1{\hbox{\vrule\vbox{\hrule{#1}\hrule}\vrule}}%
%
\def\solution{\readsol}%
\def\SolBeg{\relax}
\def\SolEnd{\relax}
\let\sol=\solution
\def\readsol#1\endsol{\soltk={#1}}%
%
\def\eightxeightpieces #1; {\piecestrue\setbox\sqbx=\vbox{\makeboard 8x8: #1; }}%
\def\fivexfivepieces #1; {\piecestrue\setbox\sqbx=\vbox{\makeboard 5x5: #1; }}%
\def\fourxfourpieces #1; {\piecestrue\setbox\sqbx=\vbox{\makeboard 4x4: #1; }}%
%
\def\beginboard{%
\numbertrue\authorfalse\origfalse\dedicfalse\stipfalse\solutiontrue\piecesfalse
\figcnttrue\condfalse
\begingroup
\initvars
\ifalphanum
	\edef\enumtok{\numbertk={\ifcase
	    \bnumber\or A\or B\or C\or D\or E\or F\or G\or H\or
		 	I\or J\or K\or L\or M\or N\or O\or P\or
			Q\or R\or S\or T\or U\or V\or W\or X\or
			Y\or Z\fi
	    }}%
\else\ifromannum
	\edef\enumtok{\numbertk={\romannumeral\bnumber}}%
\else
	\edef\enumtok{\numbertk={\number\bnumber}}%
\fi\fi
	\enumtok
\iflowcasenum
		\relax
	\else
	\edef\enumtok{\numbertk={\uppercase{\the\numbertk}}}%
	\enumtok
\fi
\let\pieces=\eightxeightpieces
}%
\def\endboard{%
  \ifpieces\relax\else\eightxeightpieces ;
  \fi
  \vtop{\parindent=0pt\baselineskip=2pt\lineskip=1pt
    %\global\bdwidth=\wd\sqbx
    \hbox{\vbox{\ifauthor
	\hbox{\box\authorbox}\vskip\baselineskip
      \else
	\ifnumber
	  \hbox to \wd\sqbx{\ifcentered\hfil\fi
	    \numberfont\the\numbertk\hfil}\fi
      \fi
      \iforig\hbox{\box\origbox}\vskip\baselineskip\fi
      \ifdedic\hbox{\box\dedicbox}\vskip\baselineskip\fi
      \henbox{\hbox{\box\sqbx}}%
    }}%
    \hbox{\vbox{%
      \vskip1pt
      \tolerance=5000\emergencystretch=10pt
      \hsize=\bdwidth
      \baselineskip=11pt
      % now set the box with stipulation, condition, and gifcnt
      \noindent
      \iffigcnt
	 %\message{we set figure counts}%
	 \setbox\figcntbx=\hbox{\remfont\hskip.7em(\the\whitecnt
	    +\the\blackcnt\ifnum\neutrcnt>0 +\the\neutrcnt n\fi)\kern-.2em}%
	 \hangindent=-\wd\figcntbx
	 \hangafter=-1
	 \rlap{\hbox to \hsize{\hfil{\box\figcntbx}}}%
      \fi
      \raggedright
      \ifstip\stipfont\the\stiptk\enspace\fi
      \ifcond\remfont\penalty-500\the\condtk\fi
    }}\vskip2pt\hbox{\box\rembox}}%
    % Write the author, and the boardnumber
    \ifsolution\immediate\write\solfd{\noexpand\ll \ifnumber\the\numbertk)\ \fi \the\authortk: \the\soltk\noexpand\eol}\fi
    %{\edef\all{\the\numbertk) \the\authortk:}%
    %  %\let\the=0\edef\iwrite{\write\solfd{\all \the\count0}}%
    %  \edef\iwrite{\write\solfd{\all \the\soltk}}%
    %  \iwrite}%
  \endgroup
  \global\advance\boardnumber by 1
}%
\def\ll #1: {\noindent\hangindent=5mm\hangafter=1 {\bf #1}:\ }%
\def\putsol{%
\immediate\closeout\solfd
\input sol\the\solnumber.tex
\global\advance\solnumber by 1
\immediate\openout\solfd=sol\the\solnumber.tex
}%
%
\def\wF{{\chessfont\ }}%
\def\bF{{\chessfont\char144}}%
\def\wwB{{\chessfont\lower .11\fontdimen2\chessfont\hbox{\char0}}}%
\def\wwS{{\chessfont\lower .11\fontdimen2\chessfont\hbox{\char1}}}%
\def\wwL{{\chessfont\lower .11\fontdimen2\chessfont\hbox{\char2}}}%
\def\swL{{\chessfont\lower .11\fontdimen2\chessfont\hbox{\char20}}}%
\def\wwT{{\chessfont\lower .11\fontdimen2\chessfont\hbox{\char3}}}%
\def\wwD{{\chessfont\lower .11\fontdimen2\chessfont\hbox{\char4}}}%
\def\wwK{{\chessfont\lower .11\fontdimen2\chessfont\hbox{\char5}}}%
\def\wsB{{\chessfont\lower .11\fontdimen2\chessfont\hbox{\char12}}}%
\def\wsS{{\chessfont\lower .11\fontdimen2\chessfont\hbox{\char13}}}%
\def\wsL{{\chessfont\lower .11\fontdimen2\chessfont\hbox{\char14}}}%
\def\ssL{{\chessfont\lower .11\fontdimen2\chessfont\hbox{\char32}}}%
\def\wsT{{\chessfont\lower .11\fontdimen2\chessfont\hbox{\char15}}}%
\def\wsD{{\chessfont\lower .11\fontdimen2\chessfont\hbox{\char16}}}%
\def\wsK{{\chessfont\lower .11\fontdimen2\chessfont\hbox{\char17}}}%
%-------------------------------------------------------------------
%
% Here comes a set of macros for parsing field-lists and generating boards
%
% We can generate board's up to at least 200 squares
%
% We use the count-registers 0 to 200 to store the type of piece going
% onto this field
%
% At first a few counterdefinitions

\countdef\xsz=200 	% Size in x(horizontal direction)
\countdef\ysz=201 	% Size in y(vertical direction)
\countdef\xysz=202	% Count of all squares
\countdef\xyp=203	% running square-number
\countdef\xp=204	% running x-number
\countdef\yp=205	% running y-number
\countdef\bn=206	% running boardnumber
\countdef\ch=207	% this is the character to set
\countdef\sum=208
\countdef\wcnt=209
\countdef\bcnt=210
\countdef\ncnt=211
%
% 
% Initialize all counters
\def\emptyboard{% We expect xsz, ysz, xysz and bn to be set properly
		% Don't use this macro in the top groups: It will overwrite
		% a lot of neccesary registers.
	\xp=0 \yp=0 \xyp=0 \sum=\bn
	% initialize all count register, one for each field
	\loop\ifnum\xyp<\xysz
		\ifodd\sum
			\count\xyp=100
		\else
			\count\xyp=144
		\fi
		\advance\xyp by 1
		\advance\xp by 1
		\advance\sum by 1
		\ifnum\xp=\xsz
			\advance\sum by -\xsz
			\xp=0
			\advance\yp by 1
			\advance\sum by 1
		\fi
	\repeat
}%
%
% now we need also routines to get the squares printed
%
\def\genrow#1{% We generate one line of the board as hbox
	   % #1 contains the number of the line wanted
	\chessfont
	\xp=0
	\xyp=#1 \multiply\xyp by \xsz
	\hbox{%
		\loop
			\ifnum\count\xyp=100
				\ 
			\else
				\char\count\xyp
			\fi
			\ifgrid\ifodd\xp\ifnum\xp<\xsz\vrule width.6pt\fi\fi\fi
			\advance\xp by1
			\advance\xyp by1
		\ifnum\xp<\xsz \repeat
	}%
}%
%
\def\genboard{% We generate the board with lines
	\vbox{
		\chessfont
		\baselineskip=0pt plus 0pt
		\lineskip=0pt plus 0pt
		\yp=\ysz
		\loop
			\advance\yp by -1
			\genrow\yp
			\ifgrid\ifodd\yp\relax\else\ifnum\yp>0\hrule height.6pt\fi\fi\fi
		\ifnum\yp>0 \repeat
	}%
}%
%
\def\initcnts#1#2#3{%
	\xsz=#1
	\ysz=#2
	\xysz=\xsz \multiply\xysz by \ysz
	\if\xysz>199\message{++++Warning: board too big: \the\xysz squares}\fi
	\bn=#3
	\wcnt=\whitecnt\bcnt=\blackcnt\ncnt=\neutrcnt
}%
%-------------------------------------------------------------------
%
% Here come the routines to parse the fieldlists
%-------------------------------------------------------------------
\def\getcolor#1{\if#1w\ch=0\let\cnt=\wcnt\else
  \if#1n\ch=6\let\cnt=\ncnt\else
  \if#1s\ch=12\let\cnt=\bcnt\else
  \error\fi\fi\fi\getpiece
}%
\def\getpiece#1{\if#1B\relax\else
  \if#1S\advance\ch by1 \else
  \if#1L\advance\ch by2 \else
  \if#1T\advance\ch by3 \else
  \if#1D\advance\ch by4 \else
  \if#1K\advance\ch by5 \else
  \if#1C\advance\ch by145 \else
  \message{error in getpiece}\fi\fi\fi\fi\fi\fi\fi
  \futurelet\rotate\checkrotate
}%
\def\checkrotate{\if\rotate u\advance\ch by 108\let\nextproc=\skipone\else
  \if\rotate l\advance\ch by 36\let\nextproc=\skipone\else
  \if\rotate r\advance\ch by 72\let\nextproc=\skipone\else
  \let\nextproc\look\fi\fi\fi\nextproc
}%
\def\skipone#1{\futurelet\whatsnext\parsefields
}%
\def\look{\futurelet\whatsnext\parsefields
}%
\def\parsefields{\if\whatsnext\eoflist\let\nextproc=\relax \else
  \let\nextproc=\sfig\fi\nextproc
}%
\def\sfig#1#2{%
	\advance\cnt by 1
	\yp=`#2\advance\yp by-`1
	\xp=`#1\advance\xp by -`a
	\xyp=\yp\multiply\xyp by \xsz\advance\xyp by \xp
	\sum=\yp\advance\sum by \xp \advance\sum by \bn
	\count\xyp=\ch
	\ifodd\sum\relax\else\advance\count\xyp by 18\fi
	\look
}%
\def\setpieces #1; {\if{#1}{}=\relax\else
  \looklist#1, \eoflist\fi
}%
\def\looklist{\futurelet\nextlist\checklist
}%
\def\checklist{\ifx\nextlist\eoflist\let\nextproc=\relax\else
  \let\nextproc=\parselist\fi\nextproc
}%
\def\parselist#1, {\getcolor#1\eoflist\looklist
}%
\def\eoflist{\relax}%
%
\def\makeboard #1x#2: #3; {%
\begingroup
\initcnts{#1}{#2}{0}\emptyboard
\if#3\relax\relax\relax\else\setpieces #3; \fi
\genboard
\global\whitecnt=\wcnt
\global\blackcnt=\bcnt
\global\neutrcnt=\ncnt
\endgroup
}%
%
% this is the end of parsing-routines for field-lists
%----------------------------------------------------------------------------

% code for testing purposes of makeboard
%\font\chessfont=chess
%\newcount\whitecnt
%\newcount\blackcnt
%\newcount\neutrcnt
%
%
%\makeboard 15x10: wBa4a6d9, wKua9;
%
%\bye
%:map q :w
:!tex genboard

%----------------------------------------
%
%\parindent=0pt
%\hrule
%\venbox{%
%\beginboard
%\author Erich Bartel;
%\orig Jugendschach;
%\stip h\#3;
%\pieces wBa2b2c2d2e2f2g2h2, sBa7b7c7d7e7f7g7h7, wTa1h1, sTa8h8,
%	wSb1g1, sSb8g8, wLc1f1, sLc8f8, wDd1, sDd8, wKe1, sKe8;
%\dedic Albert Kniest gewidmet;
%\rem a)/d4=Gnu, /\ \ \ =4,1S;
%\endboard
%}%
%\beginboard
%\orig Original;
%\stip =1;
%\rem a)/c4=2,2-S,
%	b-u)/{2,3 bis 6,7-S so, da\ss{} a2 vom Springer gedeckt ist.};
%\endboard
%\bye
%:map q :w
:!tex chess&&texx count&


%:map q :w
:!tex chess

%----------------------------------------
