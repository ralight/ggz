% This is
%		diag.ger.mf
% Es folgt eine kurze Darstellung der Moeglichkeiten der elch-macros.
% Dies ist keine komplette Dokumentation, sondern eine kommentierte
% plainTeX Quelle.
%
% Copyright (C) 1989-92  by Elmar Bartel.
%
% This program is free software; you can redistribute it and/or modify
% it under the terms of the GNU General Public License as published by
% the Free Software Foundation; either version 1, or (at your option)
% any later version.
%
% This program is distributed in the hope that it will be useful,
% but WITHOUT ANY WARRANTY; without even the implied warranty of
% MERCHANTABILITY or FITNESS FOR A PARTICULAR PURPOSE.  See the
% GNU General Public License for more details.
%
% You should have received a copy of the GNU General Public License
% along with this program; if not, write to the Free Software
% Foundation, Inc., 675 Mass Ave, Cambridge, MA 02139, USA.

\input german
\input elch
% Groesse des Diagramms angeben:
\twelvechess
% Jedes Diagramm besteht aus zwei Kuerzeln:
\beginboard
\endboard
%
% Diese beiden Anweisungen erzeugen in der Ausgabe ein leeres
% Diagramm, ohne Figuren, Autoren, Forderungen etc
%
% Innerhalb dieser beiden Kuerzel kann man weitere Kuerzel angeben,
% die Autoren, und die Quellen ueber das Diagramm setzen.
%
\beginboard
\author Erich Bartel;
\orig Jugendschach 45;
\endboard
% 
% Wichtig ist, dass nach dem Kuerzel ein Leerzeichen steht, sonst
% wird das Kuerzel nicht verstanden. Hinter dem Argument fuer das
% Kuerzel muss ein `;'Strickpunkt folgen, und dieser wiederum
% von mindestens einem Leerzeichen oder Zeilenwechsel gefolgt sein.
% Es gibt noch weitere solche Kuerzel die bekannt sind:
%
% \bnum Nummer;
%		Man kann dem Diagramm seine eigene Nummer geben,
%		wenn man nicht fortlaufende Numerierung der Diagramme
%		haben will. Wenn man bei einer anderen Nummer als 1
%		beginnen will, muss man mit \boardnumber=27 eine
%		andere Nummer (hier jetzt 27) angeben (kein `;' am
%		Ende.
%
% \dedic Widmung;
%		Hier koennen Widmungen oder etwas anderes angegeben
%		werden. Der Text erscheint unter der Quellen- und Autor-
%		angabe.
%
% Dies sind alle Kuerzel die Text oberhalb des Diagramms erscheinen
% lassen.
% Unter das Diagramm lassen sich Forderung, Bedingung und Bemerkungen
% platzieren. Dazu sind folgende Kuerzel gedacht:
%
% \stip Forderung;
%		Hier wird die Forderung angegeben.
%
% \cond Bedingung;
%		Hier wirden Bedingungen angegeben. Sie sollte nicht
%		laenger sein als das Brett breit ist.
%
% Alle bisher erwaehnten Kuerzel lassen auch mehrere Argumente zu.
% Sie sind durch ',' zu trennen. Also f"ur mehrere Autoren:
% \author Erich Bartel, Elmar Bartel;
%
% Wenn groessere Erlaeuterungen, oder Mehrlinge unter dem Diagramm
% unter dem Diagramm anzugeben sind, dann geschieht das mit dem
% Kuerzel \rem (=Remark=Bemerkung).
% Dieses Kuerzel erhaelt zwei Parameter, zum Beispiel so:
%
% \rem a)/Diagramm, b) wK\ra f4;
%		Der erste Parameter wird von dem folgenden durch
%		einen '/' getrennt. Weitere koennen durch Komma
%		getrennt folgen. Diese Bemerkungen werden unter dem
%		Diagramm gesetzt. Das geschieht so, dass alle
%		Bemerkungen in der gleichen Spalte beginnen, und die
%		groesste erste Teil einer Bemerkung die
%		Einrueckung bestimmt. 
% Ein Beispiel:
%
\beginboard
\author Erich Bartel;
\orig Original;
\stip \#5;
\rem a)/Diagramm, bb)/wBa6 nach c7;
\endboard
%
% Nun zum wichtigsten, den Figuren.
% Fuer jede Figur gibt es ein eigenes Kuerzel, das aus den ueblichen
% Abkuerzungen gebildet wird. Jeweils der Anfangsbuchstabe von
% Bauer, Springer, Laeufer, Turm, Dame, Koenig. Dieser Buchstabe
% ist gross zu schreiben. Vorangestellt wird die Farbe der Figur
% entweder w (=weiss) s (=schwarz) n (=neutral). Nachgestellt werden
% kann ein Buchstabe wenn die Figur andere Orientierung haben soll:
% l (= 90 Grad nach links gedreht), r (= 90 Grad nach rechts gedreht)
% und u (= nach unten gedreht).
% Somit wird ein neutraler Koenig mit nK bezeichnet.
% Ein solches Kuerzel wird direkt gefolgt von der Liste der Felder,
% wo diese Figur gedruckt werden soll.
% Weisser Koenig auf d4 sieht dann so aus: wKd4
% Oder mehrere schwarze Bauern: sBa3b6c7
% Die Liste der Felder hinter dem Figuren-Kuerzel wird durch ein Komma
% gefolgt von Leerzeichen abgeschlossen oder falls es die letzte
% Figur war wird das Kommando durch einen Strichpunkt abgeschlossen
% Somit sieht ein Diagramm fuer die Grundstellung folgendermassen aus:
%
\beginboard
\author Unbekannt;
\orig Unbekannt;
\pieces wBa2b2c2d2e2f2g2h2, wTa1h1, wSb1g1,
        wLc1f1, wDd1, wKe1, sBa7b7c7d7e7f7g7h7,
        sTa8h8, sSb8g8, sLc8f8, sDd8, sKe8;
\stip Forderung ?;
\cond Keine Bedingung;
\rem a)/1.Bemerkung, b)/2.Bemerkung;
\endboard
\bigskip
%
% Daneben gibt es noch drei weitere Sachen, um das Aussehen
% \gridtrue		Versieht das Diagramm mit einem Normalgitter
% \centeredtrue		Die Angaben ueber dem Diagramm werden zentriert
% \figcntfalse		Es werden keine Figurenanzahlen unter dem
%			Diagramm gedruckt.
%
%
% Hier kommen noch zwei ganz neue Sachen:
%
% Loesungen:
% Die Loesung des Problems wird eingeschlossen durch die beiden
% Kuerzel \beginsol \endsol.
% Alles was zwischen diesen beiden Kuerzeln steht wird in einer
% Datei aufgesammelt, und kann spaeter mit dem Kommando \putsol
% wieder eingelesen werden.
%
% Groesse der Diagramme kann eingestellt werden:
% 
% \twelvechess
% \elevenchess
% \tenchess
%
%
\bye
